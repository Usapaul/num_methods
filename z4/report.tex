\documentclass[12pt,a4paper]{article}
\usepackage[utf8]{inputenc}
\usepackage[english,russian]{babel}
\usepackage[OT1]{fontenc}
\usepackage{amsmath}
\usepackage{amsfonts}
\usepackage{amssymb}
\usepackage{graphicx}
\usepackage{setspace,amsmath}
\usepackage[left=2cm,right=2cm,top=2cm,bottom=2cm]{geometry}
\begin{document}

\begin{flushleft}
\large Усачев Павел, 491 группа
\end{flushleft}

\section*{Решение одномерного интегрального уравнения второго рода методом коллокаций}

\

\section{Формулировка задачи}

\begin{equation}
 u(x) - \int\limits_a^b K(x,t) u(t) dt = f(x)
\end{equation}

Требуется решить данное интегральное уравнение второго рода методом коллокаций, выбрав в качестве полиномов, по которым будет разложение, набор ортогональных полиномов.

\section{Условие задачи}

Требуется выполнить разложение по полиномом Лежандра.

Данные в условии задачи функции такие:

$$ K(x,t) = \sin (x(0.5+t^2)) \text{ \ \ -- \  ядро интегрального оператора} $$ 
$$ f(x) = x - 0.72 \text{ \ \ -- \ функция, стоящая в правой части выражения } $$

А отрезок $[a,b]$, на котором задано интегральное уравнение, по условию задачи равен отрезку $[0,1]$.

Также уравнение $(1)$ представимо с наличием в нем множителя $\alpha$ при интеграле, и по условию задачи $\alpha = 0.72$. Таким образом, можно включить значение $\alpha$ в ядро интегрального оператора и записать: $K(x,t) = 0.72 \sin (x(0.5+t^2))$. И уравнение $(1)$ с данными функциями запишется в таком виде:

$$
 u(x) - 0.72 \int\limits_0^1 \sin (x(0.5+t^2)) u(t) dt = x - 0.72
$$

\section{Вывод формул}

Для решения задачи находится приближенное решение в виде линейной комбинации ортогональных полиномов:
 
\begin{equation}
 \tilde u_n(x) = \sum \limits_{k=0}^{n-1} c_k \varphi_k(x)
\end{equation}

Где $\{ \varphi_k(x) \}_0^{\infty}$ -- набор ортогональных полиномов $k$-той степени, а за $\tilde u_n(x) $ обозначено приближенное решение задачи, для разложения которого на полиномы были взяты полиномы степени от 0 до $n-1$.

По условию, для этого разложения следует брать полиномы Лежандра. И вся суть метода коллокаций состоит в том, что нужно подобрать постоянные коэффициенты $c_k$ при полиномах разных степеней в разложении такими, чтобы на определенном наборе точек $\{ x_i \}_{i=0}^{n-1}$ (узлы коллокации) выполнялось равенство:

\begin{equation}
 \tilde u_n(x_i) - \int\limits_a^b K(x_i,t) \tilde u_n(t) dt = f(x_i)
\end{equation}
 

Подставляем в это выражение вместо $\tilde u_n(x)$ разложение по полиномам, согласно $(2)$:

$$
 \sum \limits_{k=0}^{n-1} c_k \varphi_k(x_i) - 
 \int\limits_a^b K(x_i,t) \sum \limits_{k=0}^{n-1} c_k \varphi_k(t) dt
 = f(x_i),
$$

$$
 \sum \limits_{k=0}^{n-1} c_k  
 \left (
  \varphi_k(x_i) - \int\limits_a^b K(x_i,t) \varphi_k(t) dt
 \right )
 = f(x_i)
$$

Тем самым получаем систему уравнений с матрицей $A^{n \times n}$, коэффициенты которой выражаются следующим образом: $A_{ik} = \varphi_k(x_i) - \int\limits_a^b K(x_i,t) \varphi_k(t) dt$ \\ ($i=0,...,n-1, \ k=0,...,n-1$). И система представима в виде $ AC = f(X) $, 
$ C = \left ( c_0, c_1, ..., c_{n-1} \right )^T$, 
$ f(X) = \left ( f(x_0), f(x_1), ..., f(x_{n-1}) \right )^T $.

А значения полиномов Лежандра в точках $\{ x_i \}_{i=0}^{n-1}$ получаются из следующей реккурентной формулы:

\begin{equation}
 k\varphi_k(x) = (2k-1)x \varphi_{k-1}(x) - (k-1) \varphi_{k-2}(x), \ \ 
 \varphi_0 \equiv 1, \ \ \varphi_1 = x, \ \ x \in [-1,1]
\end{equation}

Таким образом, нужно запрограммировать вычисление коэффициентов матрицы $\{ A_{ik} \}_{i,k=0}^{n-1}$ и, решив любым методом систему линейных алгебраических уравнений $ AC = f(X) $ (следует выбрать метод Гаусса решения СЛАУ с выбором ведущего элемента), получить коэффициенты $\{ c_k \}_{k=0}^{n-1}$. Вычислительная задача нахождения приближения к истинному решению интегрального уравнения на этом моменте становится решенной, и приближенное решение выражается по формуле $(2)$.

Примечание: чтобы вычислить интеграл, можно воспользоваться квадратурным методом Гаусса. Это не единственный доступный вариант, но мной был выбран именно он, причем, интеграл высчитывается по пяти точкам (см. литературу по методу интегрирования Гаусса). Потому что он обеспечивает достаточную для меня точность, и, вместо непосредственного вычисления квадратурных коэффициентов и нахождения корней полинома Лежандра, были взяты их значения из готовых таблиц коэффициентов и корней, представленных в справочной информации к методу интегрирования Гаусса.


\section{Результаты}

Данное по условию уравнение:

$$
 u(x) - 0.72 \int\limits_0^1 \sin (x(0.5+t^2)) u(t) dt = x - 0.72
$$

Найденные коэффициенты $\{ c_k \}_{k=0}^{n-1}$ для n=5 и n=7:

\begin{center}
\begin{tabular}{ccc}
$n=5$ & $n=7$ \\
\hline
-0.719705956164 & -0.720041673556 \\
 0.891282417883 &  0.891882147864 \\
 0.000365511073 & -0.000058355249 \\
 0.000832542546 &  0.001062981319 \\
 0.000071846401 & -0.000018665129 \\
                &  0.000022164609 \\
                & -0.000001981502 \\
\end{tabular}
\end{center}

\

Приближенное решение подставлено в исходное интегральное уравнение $(1)$ и были найдены значения полученной функции, а также вычислено значение невязки, равной разнице левой и правой частей в полученном выражении.

Для случая $n=5$ получены следующие значения на отрезке $[0,1]$:

\begin{center} 
\begin{tabular}{|c|c|c|}
\hline
         $x$      &      $u(x)$    &  Discrepancy  \\
\hline
    0.000000000 & -0.719999621 &  0.379163E-06 \\
\hline    
    0.100000000 & -0.631116192 & -0.368514E-06 \\
\hline    
    0.200000000 & -0.542208852 & -0.272505E-07 \\
\hline    
    0.300000000 & -0.453255551 &  0.307929E-06 \\
\hline    
    0.400000000 & -0.364232860 &  0.284680E-06 \\
\hline    
    0.500000000 & -0.275115969 &  0.111022E-15 \\
\hline    
    0.600000000 & -0.185878689 & -0.263651E-06 \\
\hline    
    0.700000000 & -0.096493454 & -0.263998E-06 \\
\hline    
    0.800000000 & -0.006931315 &  0.216073E-07 \\
\hline    
    0.900000000 &  0.082838055 &  0.269862E-06 \\
\hline    
    1.000000000 &  0.172846362 & -0.255934E-06 \\
\hline    

\end{tabular}
\end{center}

Примечание: Значение в точке $x=0.5$ получилось с ничтожно малым значением невязки, потому что это срединное значение для данного условия задачи, где функция определена на отрезке $[0,1]$, и оно соответствует фактическому нулю в вычислении корней полинома Лежандра. Вычисление невязки запрограммировано так, что при значении в середине отрезка, на котором определены полиномы Лежандра (т.е. $x=0$ на отрезке $[0,1]$) получаются одинаковые формулы для вычисления коэффициентов матрицы -- значения полиномов в точке и интеграл. Одинаковые формулы в программных вычислениях, очевидно, приводят к отсутствию разницы между получаемыми значениями. И данная невязка порядка 1e-15 вызвана только ошибками округления в процессе всех вычислений.

Ниже представлена таблица, подобная предыдущей, только для случая $n=7$, отрезок тот же -- $[0,1]$.

\begin{center} 
\begin{tabular}{|c|c|c|}
\hline
         $x$    &     $u(x)$    &   Discrepancy \\
\hline               
    0.000000000 & -0.720000002  & -0.193387E-08 \\
\hline    
    0.100000000 & -0.631115823  &  0.393881E-09 \\
\hline    
    0.200000000 & -0.542208827  & -0.184125E-08 \\
\hline    
    0.300000000 & -0.453255859  &  0.478329E-09 \\
\hline    
    0.400000000 & -0.364233144  &  0.180100E-08 \\
\hline    
    0.500000000 & -0.275115970  & -0.111022E-15 \\
\hline    
    0.600000000 & -0.185878429  & -0.174003E-08 \\
\hline    
    0.700000000 & -0.096493192  & -0.446476E-09 \\
\hline    
    0.800000000 & -0.006931336  &  0.166027E-08 \\
\hline    
    0.900000000 &  0.082837784  & -0.343064E-09 \\
\hline    
    1.000000000 &  0.172846618  &  0.162673E-08 \\
\hline
\end{tabular}
\end{center}


\end{document}

