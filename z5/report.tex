\documentclass[12pt,a4paper]{article}
\usepackage[utf8]{inputenc}
\usepackage[english,russian]{babel}
\usepackage[OT1]{fontenc}
\usepackage{amsmath}
\usepackage{amsfonts}
\usepackage{amssymb}
\usepackage{graphicx}
\usepackage{setspace,amsmath}
\usepackage[left=2cm,right=2cm,top=2cm,bottom=2cm]{geometry}
\begin{document}

\begin{flushleft}
\large Усачев Павел, 491 группа
\end{flushleft}

\section*{Некорректная задача: решение одномерного интегрального уравнения первого рода}


\

\section{Формулировка задачи}

\begin{equation}
 \int\limits_a^b K(x,t) u(t) dt = f(x)
\end{equation}

Требуется решить данное интегральное уравнение первого рода методом Ритца. Координатные функции -- полиномы Лежандра.

\section{Условие задачи}

Требуется выполнить разложение по полиномом Лежандра.

Данные в условии задачи функции такие:

$$ K(x,t) = \ln (2 + xt) \text{ \ \ -- \  ядро интегрального оператора} $$ 
$$ f(x) = 0.5\frac{2+x}{2-x} \ln \left ( \frac{4}{x+2} \right )
  \text{ \ \ -- \ функция, стоящая в правой части выражения } 
$$

А отрезок $[a,b]$, на котором задано интегральное уравнение, по условию задачи равен отрезку $[0,1]$.



\section{Вывод формул}

Для решения задачи находится приближенное решение в виде линейной комбинации ортогональных полиномов:
 
\begin{equation}
 \tilde u_n(x) = \sum \limits_{k=0}^{n-1} c_k \varphi_k(x)
\end{equation}

Где $\{ \varphi_k(x) \}_0^{\infty}$ -- набор ортогональных полиномов $k$-той степени, а за $\tilde u_n(x) $ обозначено приближенное решение задачи, для разложения которого на полиномы были взяты полиномы степени от 0 до $n-1$.

\

Пусть дано гильбертово пространство $H$, где задано скалярное произведение $(u,v)$.

$Au=f$ - симметрично и положительно определено, при условии выполнения следующих равенства и неравенства:

$$(Au,v) = (u,Av), \ \ (Au,v) \geq \alpha(u,v)$$

Рассмотрим функционал:
 
$$F(u) = (Au,v)-2(f,u)$$

Ограниченность оператора означает, что энергетическое пространство -- это исходное $H$. Его min существует и достигается на единственном элементе, который есть решение уравнения:\\

$$\min f(u) = f(u^*), \ \ Au^*=f$$

Минимальный элемент:

$$u^* \in H: (u^*,v)_A = (f,v) \ \ \forall v \in H$$


Рассмотрим координатную систему и линейную комбинацию в ней. Множество всех линейных комбинаций составит линейное подпространство:

$$ \left( \varphi_k \right)^n \qquad \left({\sum\limits_{k=1}^n C_k \varphi_k} \right)= H_n \qquad \min_{H_n} F(u)$$
$$u^*_n \in H: (u^*_n,v_n)_A = (f,v_n)  \qquad \forall v_n \in H$$

\

Если он принадлежит $H_n$, то:
$$u^*_n = {\sum\limits_{k=1}^n C_k \phi_k}$$
$$\left( {\sum C_k \phi_k}, v_n\right)_A = \left( f, v_n \right)$$

Достаточное условие, требуемое для линейной комбинации:
$$ \left({\sum\limits_{k=1}^n C_k \phi_k}, \phi_m \right)_A =  \left( f, \phi_m \right)$$
$$ {\sum\limits_{k=1}^n (\phi_k, \phi_m)_A} = (f, \phi_m)$$
$${\sum\limits_{k=1}^n (A\phi_k,\phi_m)}\cdot C_k = (f, \phi_m)$$

\

Добавляем регуляризатор к исходному интегральному уравнению:

Составим матрицу $A = \alpha I + K$.

$$Au = u +\int\limits_a^b K(x,t)u(t)dt$$

По условию, для этого разложения следует брать полиномы Лежандра. И вся суть метода состоит в том, что нужно подобрать постоянные коэффициенты $c_k$ при полиномах разных степеней в разложении такими, чтобы на определенном наборе точек $\{ x_i \}_{i=0}^{n-1}$ (узлы сетки) выполнялось матричное равенство со следующими коэффициентами:

$$(A\phi_k, \phi_m) = \alpha \int\limits_a^b \phi_k(x)\phi_m(x)dx + \int\limits_a^b K(x,t) \phi_k(t) \phi_m(x)dx = a_{mk}$$
$$ m = 1, ..., n \qquad k = 1, ... , n $$

\

И система представима в виде $ AC = B $, 
$ C = \left ( c_0, c_1, ..., c_{n-1} \right )^T$, 

$ B = \left ( b_0, b_1, ..., b_{n-1} \right )^T $, 
где $\{ b_i \}_{i=0}^{n-1}$ выражаются следующим образом:

$$b_m = \int\limits_a^b f(x)\phi_m(x)dx$$ 

\

А значения полиномов Лежандра в точках $\{ x_i \}_{i=0}^{n-1}$ получаются из следующей реккурентной формулы:

\begin{equation}
 k\varphi_k(x) = (2k-1)x \varphi_{k-1}(x) - (k-1) \varphi_{k-2}(x), \ \ 
 \varphi_0 \equiv 1, \ \ \varphi_1 = x, \ \ x \in [-1,1]
\end{equation}

\

Таким образом, нужно запрограммировать вычисление коэффициентов матрицы $\{ A_{ik} \}_{i,k=0}^{n-1}$ и, решив любым методом систему линейных алгебраических уравнений $ AC = B $ (следует выбрать метод Гаусса решения СЛАУ с выбором ведущего элемента), получить коэффициенты $\{ c_k \}_{k=0}^{n-1}$. Вычислительная задача нахождения приближения к истинному решению интегрального уравнения на этом моменте становится решенной, и приближенное решение выражается по формуле $(2)$.

Примечание: чтобы вычислить интеграл, можно воспользоваться квадратурным методом Гаусса. Это не единственный доступный вариант, но мной был выбран именно он, причем, интеграл высчитывается по пяти точкам (см. литературу по методу интегрирования Гаусса). Потому что он обеспечивает достаточную для меня точность, и, вместо непосредственного вычисления квадратурных коэффициентов и нахождения корней полинома Лежандра, были взяты их значения из готовых таблиц коэффициентов и корней, представленных в справочной информации к методу интегрирования Гаусса.


\section{Результаты}

Данное по условию уравнение:


$$ \alpha u + \int\limits_a^b K(x,t) u(t) dt = f(x) $$

$$ K(x,t) = \ln (2 + xt) \text{ \ \ -- \  ядро интегрального оператора} $$ 
$$ f(x) = 0.5\frac{2+x}{2-x} \ln \left ( \frac{4}{x+2} \right )
  \text{ \ \ -- \ функция, стоящая в правой части выражения } 
$$


Найденные коэффициенты $\{ c_k \}_{k=0}^{n-1}$ для n=5 и n=7 и $\alpha=0.1$:

\begin{center}
\begin{tabular}{ccc}
$n=5$ & $n=7$ \\
\hline

 0.484356807121 &  0.575225404029 \\
-0.095093700564 & -0.306947432210 \\
 0.016299779976 &  0.283766892447 \\
-0.002163232753 & -0.273607279290 \\
 0.000193615759 &  0.204050624058 \\
                & -0.102290387588 \\
                &  0.030421476558 \\
\end{tabular}
\end{center}


\

Приближенное решение подставлено в исходное интегральное уравнение $(1)$ и были найдены значения полученной функции, а также вычислено значение невязки, равной разнице левой и правой частей в полученном выражении.

Для случая $n=5$ получены следующие значения на отрезке $[0,1]$:

\begin{center} 
\begin{tabular}{|c|c|c|}
\hline
         $x$      &      $u(x)$    &  Discrepancy  \\
\hline
    0.000000000 & 0.468250643 & -0.212417E-06  \\
\hline
    0.100000000 & 0.459692251 &  0.818888E-07  \\
\hline
    0.200000000 & 0.451705123 &  0.302292E-07  \\
\hline
    0.300000000 & 0.444242916 & -0.518506E-07  \\
\hline
    0.400000000 & 0.437263008 & -0.570600E-07  \\
\hline
    0.500000000 & 0.430726492 & -0.476668E-32  \\
\hline
    0.600000000 & 0.424598177 &  0.536876E-07  \\
\hline
    0.700000000 & 0.418846594 &  0.458964E-07  \\
\hline
    0.800000000 & 0.413443986 & -0.251662E-07  \\
\hline
    0.900000000 & 0.408366318 & -0.640921E-07  \\
\hline
    1.000000000 & 0.403593270 &  0.156212E-06  \\
\hline

\end{tabular}
\end{center}

Примечание: Значение в точке $x=0.5$ получилось с ничтожно малым значением невязки, потому что это срединное значение для данного условия задачи, где функция определена на отрезке $[0,1]$, и оно соответствует фактическому нулю в вычислении корней полинома Лежандра. Вычисление невязки запрограммировано так, что при значении в середине отрезка, на котором определены полиномы Лежандра (т.е. $x=0$ на отрезке $[0,1]$) получаются одинаковые формулы для вычисления коэффициентов матрицы -- значения полиномов в точке и интеграл. Одинаковые формулы в программных вычислениях, очевидно, приводят к отсутствию разницы между получаемыми значениями. И данная невязка порядка 1e-15 вызвана только ошибками округления в процессе всех вычислений.

Ниже представлена таблица, подобная предыдущей, только для случая $n=7$, отрезок тот же -- $[0,1]$.

\begin{center} 
\begin{tabular}{|c|c|c|}
\hline
         $x$    &     $u(x)$    &   Discrepancy \\
\hline
    0.000000000 & 0.465087659 & -0.316511E-03 \\
\hline    
    0.100000000 & 0.461110263 &  0.141883E-03 \\
\hline    
    0.200000000 & 0.452305852 &  0.601031E-04 \\
\hline    
    0.300000000 & 0.443073238 & -0.117020E-03 \\
\hline    
    0.400000000 & 0.435815142 & -0.144844E-03 \\
\hline    
    0.500000000 & 0.430726492 &  0.144445E-33 \\
\hline    
    0.600000000 & 0.426283636 &  0.168600E-03 \\
\hline    
    0.700000000 & 0.420434470 &  0.158833E-03 \\
\hline    
    0.800000000 & 0.412489464 & -0.954774E-04 \\
\hline    
    0.900000000 & 0.405713615 & -0.265334E-03 \\
\hline    
    1.000000000 & 0.410619298 &  0.702759E-03 \\
\hline
\end{tabular}
\end{center}


\end{document}

