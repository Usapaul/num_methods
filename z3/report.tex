\documentclass[12pt,a4paper]{article}
\usepackage[utf8]{inputenc}
\usepackage[english,russian]{babel}
\usepackage[OT1]{fontenc}
\usepackage{amsmath}
\usepackage{amsfonts}
\usepackage{amssymb}
\usepackage{graphicx}
\usepackage{setspace,amsmath}
\usepackage[left=2cm,right=2cm,top=2cm,bottom=2cm]{geometry}
\begin{document}

\begin{flushleft}
\large Усачев Павел, 491 группа
\end{flushleft}

\section*{Определение наибольшего по модулю собственного числа матрицы при помощи последовательных итераций}

\

\section{Формулировка задачи}
Для данной матрицы при помощи степенного метода определить наибольшее и наименьшее по модулю собственные числа и соответствующие им собственные векторы. А также, применяя метод Гивенса, найти все собственные числа данной матрицы.

\section{Условие задачи}

Дана следующая симметричная матрица порядка $ s \times s $, записанная в виде условий для заполнения значений на диагонали, на под- и наддиагонали, и всех остальных чисел:

$$ a_{i,i} = a \ \ \text{($a$ -- const)} $$
$$ a_{i,i+1} = \left(\frac{i}{s}-\frac{1}{2}\right)b \ \ \text{($b$ -- const)}$$
$$ a_{i,k} = \frac{4}{(i+k)^2}, \ \ |i-k|>1 $$

\

Мною были взяты значения $a=3.0, \ b=1.0$, однако, программа, выполняющая вычисления, может непосредственно с экрана просить ввести значения. По условию данные константы могут быть любыми. Стоит заметить, что увеличение значений этих констант и подбор плохой их комбинации может увеличить количество итераций, необходимых для сходимости метода, потому что непосредственно меняются собственные числа матрицы. Близкие значения собственных чисел также замедляют поиск всех корней по методу Гивенса.

\section{Вывод формул}

\subsection{Наибольшее по модулю собственное число}

Пусть дана $A$ -- матрица порядка $s$. $X$ -- произвольный вектор, $\{\lambda_i\}_{i=1}^s$ -- собственные числа матрицы $A$. Занумеруем собственные числа в порядке их убывания по модулю, предполагая, что существует наибольшее по модулю собственное число, т.е. существует такая нумерация собственных чисел, что $|\lambda_1| > |\lambda_2| \geq ... \geq |\lambda_s|$. Также утверждаем, что исследуемая матрица $A$ обладает полным набором собственных векторов:

$$ u_i \ (i=1,...,s): \ Au_i = \lambda_i u_i $$

Рассмотрим следующую последовательность векторов:

$$ X, \ AX, \ A(AX)=A^2X, \ ... \ , A^nX=A(A^{n-1}X), \ ...  $$

\newpage

Разложим по базису $\{u_i\}_{i=1}^s$ каждый вектор из последовательности:

$$ X  =  c_1 u_1 + c_2 u_2 + ... + c_s u_s, $$
$$\ \ \ \ \ \ \, AX=c_1\lambda_1u_1 + c_2\lambda_2u_2 + ... + c_s\lambda_su_s, $$
$$
\ \ \ \ \ \
A^nX = c_1 \lambda_1^n u_1 + c_2 \lambda_2^n u_2 + ... + c_s \lambda_s^n u_s.
$$


При достаточно больших $n$ первый член суммы в разложении векторов будет преобладать над остальными, и тогда $A^nX \approx c_1 \lambda_1^n u_1$.

При вычислении каждого следующего приближения $x_{n+1}$ будем выполнять нормировку на первую его компоненту: 

$$ \tilde X_{n+1} = AX_n, \ X_{n+1} = \frac{\tilde X_{n+1}}{\{\tilde X_{n+1}\}_1} $$

Таким образом, если $ X_n \approx u_1 $, то $ AX_n \approx \lambda_1 u_1 $ и $\{\tilde X_{n+1}\}_1 \approx \lambda_1 $, и такими последовательными приближениями мы получаем собственное число $ \lambda_1 $, которое по разложению является наибольшим по модулю среди всех собственных чисел матрицы $A$. 

Этот алгоритм и называется степенным методом. Осталось только его перевести на программый язык. Это было успешно сделано. Необходимо также задать вектор начальных приближений $X_0$. Также была проведена проверка равенства $Au_1 = \lambda_1 u_1$

\subsection{Наименьшее по модулю собственное число}

Нахождение наименьшего по модулю собственного числа матрицы $A$ осуществляется методом сдвига. Суть его такова:
\begin{itemize}
\item определяем матрицу $\hat A = A-\lambda_1 E$.
\item задаем начальный вектор приближений $X_0$
\item выполняем процедуру, настроенную на поиск наибольшего по модулю собственного числа матрицы $A$, только подаем ей на вход матрицу $\hat A$
\item к полученному в результате работы процедуры числу $\lambda_0$ прибавляем уже найденное максимальное по модулю собственное число $\lambda_1$
\item полученное число $\lambda = \lambda_0 + \lambda_1$ -- наименьшее по модулю собственное число данной матрицы $A$
\end{itemize}

\subsection{Нахождение всех собственных чисел}

Нахождение всех собственных чисел данной матрицы $A$ будет осуществляться согласно методу вращений Гивенса. Он заключается в том, что исходная матрица преобразовывается к трёхдиагональному виду. Для его описания следует ввести понятие матрицы вращения.

Элементарной матрицей вращения $T_{ij}$ называется единичная матрица $E$, в которой изменены четыре элемента: вместо единиц в позициях $(p,p)$ и $(q,q)$ поставлено число $c$, в позиции $(p,q)$ поставлено число $s$, а в позиции $(q,p)$ поставлено $-s$:

$$
 \begin{array}{ccccccccc}
 \ & \ & \ \ \ \ & p & \ \ \ \ \ & q & & & 
 \end{array}
$$ 
$$
 T_{ij} =
 \left ( 
  \begin{array}{ccccccccc}	
   1 & . & . & . & . & . & . & . & 1 \\
   \vdots & \ddots & & & & & & & \vdots \\
   & & 1  \ \\
   . & . & . & c & \ldots & s & . & . & . \\
   & & & \vdots & \ddots & \vdots & & & \\
   . & . & . & -s & \ldots & c & . & . & . \\
    &  &  &  &  &  & 1 &  &  \\
   \vdots & & & & & & & \ddots & \vdots \\
   1 & . & . & . & . & . & . & . & 1 \\
  \end{array}   
 \right ) 
 \begin{array}{c}
  \\
  \\
  \\
  p \\
  \\
  q \\
  \\
  \\
  \ 
 \end{array}
$$

\

При этом числа $c$ и $s$ удовлетворяют соотношению: 
$$ c^2 + s^2 = 1 $$

Метод Гивенса относится к прямым методам. Метод Гивенса позволяет за $N$ шагов привести матрицу $A$ к трехдиагональному виду, если $A$ -- симметричная, либо к левому почти треугольному виду, если $A$ -- произвольная квадратная матрица. Этот метод основан на подобных преобразованиях исходной матрицы с помощью матриц вращения:

$$ A_{k+1} = T_{ij}^T \cdot A_k \cdot T_{ij} $$

Матрицы вращения являются ортогональными матрицами. Пусть 
$$ T = T_{23} \cdot T_{24} \cdot ... \cdot T_{2n} \cdot ... \cdot T_{n-1,n},$$
Тогда можно записать: $ A_N = T^T \cdot A \cdot T $, где $T$ - ортогональная матрица. Это означает, что матрица $A_N$ подобна $A$ и имеет те же собственные значения. Следовательно, для исходной матрицы будет также решена проблема собственных значений.

Для приведенной таким образом симметричной матрицы к трехдиагональной, методом бисекции можно решить полную или частичную проблему собственных значений. 

Рассмотрим квадратную матрицу $A = \{ a_{ij} \}$ порядка $n$. Пусть $T_{ij}$ -- матрица вращения порядка $n$ с элементами $c$ и $s$ такими, что $c^2 + s^2 = 1$. 

Построим последовательность из подобных матриц, с начальной матрицей $A$, такую, что у последней матрицы в позициях $(1,3), (1,4), ..., (1,n), (2,4), (2,5), ..., (2, n), ..., (n-2,n)$ стояли бы нули. Другими словами, все элементы последней матрицы в позициях $(i, j)$ таких, что $i + 1 < j$ были бы нули, а последняя матрица была трехдиагональной (матрица $A$ симметричная). Число матриц последовательности равно числу обнуляемых элементов. Последнюю матрицу такой последовательности условно можно записать так:

$$
 S = 
 (
  T_{n-1,n}^T \cdot ... \cdot T_{2n}^T \cdot ... \cdot T_{24}^T \cdot T_{23}^T
 ) 
 \cdot A \cdot
 (T_{23} \cdot T_{24} \cdot ... \cdot T_{n-1,n})
$$

В этой записи присутствуют все матрицы набора.

Рассмотрим как вычислить эти матрицы и выбрать элементы матрицы вращения $c$ и $s$ на каждом шаге. Рассмотрим первое подобное преобразование и найдем вторую матрицу последовательности:

$$
 G = T_{i,j}^T \cdot A \cdot T_{i,j}, \ i=2, \ j=3
$$

Пусть $B = A \cdot T_{i,j}$. Тогда все столбцы матрицы $A$ совпадают со столбцами матрицы $B = \{ b_{ij} \} $ за исключением $i$, $j$ столбцов.

$$
 g_{i-1,j} = b_{i-1,j} = sa_{i-1,i} + ca_{i-1,j} = 0
$$

Отсюда с учетом $c^2 + s^2 = 1$ находим:

$$
 c = \frac{a_{i-1,i}}{\sqrt{a_{i-1,i}^2 + a_{i-1,j}^2}}, \ 
 s = - \frac{a_{i-1,j}}{\sqrt{a_{i-1,i}^2 + a_{i-1,j}^2}}
$$

Подставим найденные элементы $c$ и $s$ в матрицу вращения и получим вторую матрицу последовательности $G$.

Аналогично проводим следующее умножение и получаем третью матрицу

$$
 G_1 = T_{i,j}^T \cdot G \cdot T_{i,j}, i=2, \ j=4
$$ 

Продолжая подобным образом, получим последнюю матрицу $S$, обладающую нужными свойствами.

Наконец, нужно найти корни многочлена, задающего минор трехдиагональной матрицы (пояснение см. в соответствующей литературе или в комментариях в коде моей программы). Корни полинома находятся методом бисекций. Также применяется метод Ньютона, когда найдено достаточно хорошее приближение к корню.

\section{Результаты}

\

$$ s=6, \ a_{const}=3.0, \ b_{const}=1.0 $$

\

$$ \lambda_{max} = 3.5604910501129028096 $$
$$ \text{Число итераций: } 451 $$
$$ \text{Невязка: } 3.3831760041282164718\text{E}-010 $$

\

$$ \lambda_{min} = 2.5573659171286642238 $$
$$ \text{Число итераций: } 219 $$
$$ \text{Невязка: } 8.9680290220893873202\text{E}-011 $$

\

Все собственные числа, найденные методом Гивенса:

\

\begin{center}


\begin{tabular}{cc}
 & $\lambda$ \\
 \hline
 $\lambda_{max}$  &  3.5604910500 \\
 2  &  3.4072806935 \\
 3  &  3.0022216826 \\
 4  &  2.8271073069 \\
 5  &  2.6455333498 \\
 $\lambda_{min}$  &  2.5573659171
\end{tabular}

\end{center}





\end{document}

