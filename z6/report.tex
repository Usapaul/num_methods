\documentclass[12pt,a4paper]{article}
\usepackage[utf8]{inputenc}
\usepackage[english,russian]{babel}
\usepackage[OT1]{fontenc}
\usepackage{amsmath}
\usepackage{amsfonts}
\usepackage{amssymb}
\usepackage{graphicx}
\usepackage{setspace,amsmath}
\usepackage[left=2cm,right=2cm,top=2cm,bottom=2cm]{geometry}
\begin{document}

\begin{flushleft}
\large Усачев Павел, 491 группа
\end{flushleft}

\section*{Решение одномерного уравнения теплопроводности методом сеток}

\

\section{Формулировка задачи}

Уравнение теплопроводности в общем виде:

$$
 \frac{\partial u}{\partial t} =
 a(x,t) \frac{\partial ^2 u}{\partial x^2} +
 b(x,t) \frac{\partial u}{\partial x} +
 c(x,t) u(x,t) + f(x,t),
$$
$$
 x \in [a,b], \ t \in [0,T]
$$

Его мы и хотим решить методом сеток

Начальные условия:

$$
 u(x,0) = \phi(x)
$$

Граничные условия:

$$
 \left[\alpha_1 \frac{\partial u}{\partial x} + \alpha_2 u\right]_{x=a}=\Psi_1(t) 
$$
$$
 \left[\beta_1 \frac{\partial u}{\partial x} + \beta_2 u\right]_{x=b} = \Psi_2(t) 
$$

Шаг сетки по $x$ при заданном $n$: $ \displaystyle h = \frac{b-a}{n}. \ $
$ \displaystyle x_i = a + ih, \ i=0,...,n $

При условии устойчивости:
$$
 \frac{\tau}{h^2} \leq \frac{1}{2A}, \ 
 A = \mathop{\text{max}}_{\Omega} |a(x,t)|, \ 
 \text{где $\Omega$ -- область существования решения $u(x,t)$}
$$

Разбиение по $t$ выглядит следующим образом:

$$
 t_k = k \cdot \tau, \ k=1,..,m, \ 
 \text{$m$ -- количество {\bfseries слоев}, которое нас интересует}
$$

\

Решение на сетке:
$ \displaystyle u_i^k = u(x_i,t_k) $

\section{Условие задачи}

$$
 \frac{\partial u}{\partial t} =
 \frac{\partial ^2 u}{\partial x^2} + \exp(-t)(x^2 - x + 2) 
$$
$$
 x \in [0,1], \ t \in [0,T]
$$

\

Начальные условия: $ \displaystyle u(x,0) = \cos 0.5x + (1-x)x$

\

Граничные условия:
$$ u(0,t) = \exp(-0.25t) $$
$$ u(1,t) = \exp(-0.25t) \cos 0.5 $$

Причем, известно точное решение задачи:
$$
 \displaystyle u^*(x,t) = 
 \exp(-0.25t) \cos 0.5x + \exp(-t) (1-x)x
$$

\section{Вывод формул}

\subsection{Явный метод}

Каждая производная заменяется приближающим ее сеточным выражением. И тогда получается:


\begin{equation} 
 \frac{u_i^{k+1} - u_i^k}{\tau} = 
 a_{ik} \frac{u_{i+1}^k - 2u_i^k + u_{i-1}^k}{h^2} +
 b_{ik} \frac{u_{i+1}^k-u_{i-1}^k}{2h} + 
 c_{ik} u_i^k + f_{ik}
\end{equation}

В моей задаче $a(x,t) \equiv 1, \ b(x,t) \equiv 0, \ c(x,t) \equiv 0, \ $
$ f(x,t) = \exp(-t)(x^2 - x + 2)  $

Приведем подобные слагаемые в исходном сеточном уравнении $(1)$ и получим уравнение следующего вида:

\begin{equation}
 u_i^{k+1} = A_{ik} u_{i+1}^k - B_{ik} u_i^k + C_{ik} u_{i-1}^k + D_{ik}
\end{equation}


Коэффициенты $A_{ik},\ B_{ik},\ C_{ik},\ D_{ik}\ $ пока не знаем

Дальше приводятся выкладки и приведение подобных слагаемых в уравнении $(1)$, соответственно будут найдены коэффициенты $A_{ik},\ B_{ik},\ C_{ik},\ D_{ik}\ $:

$$
 u_i^{k+1} \cdot \frac{1}{\tau} + u_i^k \cdot \frac{-1}{\tau} =
 u_{i+1}^k \frac{a_{ik}}{h^2} + \frac{-2a_{ik}}{h^2} u_i^k + 
 \frac{a_{ik}}{h^2} u_{i-1}^k + \frac{b_{ik}}{2h} u_{i+1}^k +
 \frac{-b_{ik}}{2h} u_{i-1}^k + c_{ik} u_i^k + f_{ik}
$$

$$
 u_i^{k+1} \cdot \frac{1}{\tau} =
 \left ( 
  \frac{a_{ik}}{h^2} + \frac{b_{ik}}{2h}
 \right )
 u_{i+1}^k
 +
 \left (
  \frac{1}{\tau} - \frac{2a_{ik}}{h^2} + c_{ik}
 \right )
 u_i^k
 +
 \left (
  \frac{a_{ik}}{h^2} - \frac{b_{ik}}{2h}
 \right )
 u_{i-1}^k
 + f_{ik}
$$

$$
 A_{ik} = \tau \left (  \frac{a_{ik}}{h^2} + \frac{b_{ik}}{2h} \right ),
$$
$$
 B_{ik} = -\tau  \left ( \frac{1}{\tau} - \frac{2a_{ik}}{h^2} + c_{ik} \right )
 = -1 + \left ( \frac{2a_{ik}}{h^2} - c_{ik} \right ) \tau,
$$
$$
 C_{ik} = \tau \left (  \frac{a_{ik}}{h^2} - \frac{b_{ik}}{2h} \right ), \ \ 
 D_{ik} = \tau f_{ik}
$$

Теперь, зная коэффициенты в уравнении $(2)$, можно приступить к непосредственному вычислению $u_i^{k+1}$ для $k=0,...,m-1$. Соответственно это будет цикл. Заметим, что каждое значение сеточного решения вычисляется через три значения предыдущего слоя: в той же точке по $x$ и в двух соседних. У краевых точек нет соседних, но значение функции в краевых точках нам и так известно, так как заданы краевые условия.

Также нужно с чего-то начинать: чтобы вычислить значения функции нового слоя через значения предыдущего слоя, надо, чтобы весь предыдущий слой был заполнен. При наличии начальных условий сразу при старте цикла в программе нулевой слой оказывается уже заполненным, что обеспечивает дальнейший ход цикла.

Таким образом, по явному методу значения в точках сетки вычислены, и это мы и должны принять за приближение к истинному решению задачи. В рамках численных методов задача решена. Теперь следует разобраться с неявным методом.

\subsection{Неявный метод}

Каждая производная заменяется приближающим ее сеточным выражением. И тогда получается:

\begin{equation} 
 \frac{u_i^k - u_i^{k-1}}{\tau} = 
 a_{ik} \frac{u_{i+1}^k - 2u_i^k + u_{i-1}^k}{h^2} +
 b_{ik} \frac{u_{i+1}^k-u_{i-1}^k}{2h} + 
 c_{ik} u_i^k + f_{ik}
\end{equation}

Приведем подобные слагаемые в исходном сеточном уравнении $(3)$ и получим уравнение следующего вида:

\begin{equation}
 \tilde A u_{i+1}^k - \tilde B_{ik} u_i^k + \tilde C_{ik} u_{i-1}^k +
 \tilde D_{ik} = u_i^{k-1}
\end{equation}

Коэффициенты $\tilde A_{ik},\ \tilde B_{ik},\ \tilde C_{ik},\ \tilde D_{ik}\ $ пока не знаем

Дальше приводятся выкладки и приведение подобных слагаемых в уравнении $(1)$, соответственно будут найдены коэффициенты $\tilde A_{ik},\ \tilde B_{ik},\ \tilde C_{ik},\ \tilde D_{ik}\ $:

$$
 u_i^k \cdot \frac{1}{\tau} + u_i^{k-1} \cdot \frac{-1}{\tau} =
 u_{i+1}^k \frac{a_{ik}}{h^2} + \frac{-2a_{ik}}{h^2} u_i^k + 
 \frac{a_{ik}}{h^2} u_{i-1}^k + \frac{b_{ik}}{2h} u_{i+1}^k +
 \frac{-b_{ik}}{2h} u_{i-1}^k + c_{ik} u_i^k + f_{ik}
$$

$$
 \left ( 
  \frac{a_{ik}}{h^2} + \frac{b_{ik}}{2h}
 \right )
 u_{i+1}^k
 +
 \left (
  \frac{1}{\tau} + \frac{2a_{ik}}{h^2} - c_{ik}
 \right )
 u_i^k
 +
 \left (
  \frac{a_{ik}}{h^2} - \frac{b_{ik}}{2h}
 \right )
 u_{i-1}^k
 + f_{ik}
 = u_i^{k-1} \cdot \frac{-1}{\tau}
$$

$$
 \tilde A_{ik} = -\tau \left (  \frac{a_{ik}}{h^2} + \frac{b_{ik}}{2h} \right )
 = - A_{ik},
$$
$$
 \tilde B_{ik} = -\tau  
 \left ( 
  \frac{1}{\tau} + \frac{2a_{ik}}{h^2} - c_{ik} 
 \right ) 
 = -1 - \left ( \frac{2a_{ik}}{h^2} - c_{ik} \right ) \tau
 = -1 - (B_{ik} + 1) = -B_{ik} - 2,
$$
$$
 \tilde C_{ik} = -\tau \left (  \frac{a_{ik}}{h^2} - \frac{b_{ik}}{2h} \right )
 = -C_{ik}, \ \ 
 \tilde  D_{ik} = -\tau f_{ik} = -D_{ik}
$$

Таким образом, с моими условиями задачи получилась следующая система:

$$
 \begin{cases}
  u_0^k = \Psi_1(t_k) 
  \\
  \tilde A u_{i+1}^k - \tilde B_{ik} u_i^k + \tilde C_{ik} u_{i-1}^k +
  \tilde D_{ik} = u_i^{k-1}, & i=1,...,n-1 
  \\ 
  u_n^k = \Psi_2(t_k)
 \end{cases}
$$

Для всех $k=1,...,m$

\

Получилась система с трехдиагональной матрицей. Данную систему следует решать методом прогонки. В программе к этой задаче была использована процедура, созданная для решения первой задачи по численным методам. См. отчет к ней. В соответствии с принятыми там обозначениями нужно принять:

$$ a_{ik} = \tilde A_{ik}, \ b_{ik} = \tilde B_{ik}, \ c_{ik} = \tilde C_{ik}, $$
$$  \ g_{ik} = -\tilde D_{ik} + u_i^{k-1} $$
$$ \kappa_1 = 0, \ \nu_1 = \Psi_1(t_k) $$
$$ \kappa_2 = 0, \ \nu_2 = \Psi_2(t_k) $$

Результат работы процедуры -- вектор решений системы $\{ u_i^k \}_{i=0}^n$ для конкретного $k$. В вышеупомянутом цикле $k$ пробегает все значения от 1 до $m$ ($u_i^0$ известно из начальных условий). Таким образом, двумерная сетка полностью заполнена, т.е. получено решение задачи в виде численного приближения к истинному решению. 


\section{Результаты}

\

Так как для моей задачи известно точное решение в виде аналитически заданной функции, можно напрямую сравнивать получаемые в узлах сетки значения функции, получаемые путем вычислений в программе методом сеток, со значениями, вычисляемые непосредственно из формулы истинного решения.

Было проверено, что значения действительно совпадают теоретически обоснованной точностью при заявленных параметрах разбиения сетки. Для иллюстрации ниже приведена таблица значений невязки (норма разности двух векторов: вычисленное значение аналитически заданной функции -- точного решения; и получаемые программой табличные значения приближения к решению, даваемые в результате работы вышеописанных алгоритмов) для $n=100, \ m=1000$. 


\begin{center}

\begin{tabular}{|c|c|c|}
 \hline
 & \multicolumn{2}{|c|}{Discrepancy} \\
 \cline{2-3}
 % \raisebox{x}[a][b] помещает текст на уровень x в терминах номеров строк
 % и еще дополнительное смещение вверх/вниз на a/b -- но вряд ли все так тупо
 \raisebox{1.5ex}[0cm][0cm]{layer}
 & explicit method & implicit method \\
\hline
   0 & 0.00000000     & 0.00000000     \\
\hline
  50 & 3.14269687E-06 & 2.88440915E-05 \\
\hline
 100 & 2.50835728E-06 & 5.70365992E-05 \\
\hline
 150 & 2.59810622E-06 & 8.34867897E-05 \\
\hline
 200 & 3.54133863E-06 & 1.08988832E-04 \\
\hline
 250 & 4.31670969E-06 & 1.36303788E-04 \\
\hline
 300 & 5.53393102E-06 & 1.60818483E-04 \\
\hline
 350 & 5.66400968E-06 & 1.87064506E-04 \\
\hline
 400 & 8.85725785E-06 & 2.09875900E-04 \\
\hline
 450 & 7.43721739E-06 & 2.35277141E-04 \\
\hline
 500 & 9.45086140E-06 & 2.57606764E-04 \\
\hline
 550 & 8.92060689E-06 & 2.83729838E-04 \\
\hline
 600 & 1.22891133E-05 & 3.06137837E-04 \\
\hline
 650 & 9.63495768E-06 & 3.32321273E-04 \\
\hline
 700 & 1.14461091E-05 & 3.54934891E-04 \\
\hline
 750 & 1.19720789E-05 & 3.79042758E-04 \\
\hline
 800 & 1.31726265E-05 & 4.00894292E-04 \\
\hline
 850 & 1.30291719E-05 & 4.23227291E-04 \\
\hline
 900 & 1.28739130E-05 & 4.45596030E-04 \\
\hline
 950 & 1.38526630E-05 & 4.66498255E-04 \\
\hline
1000 & 1.26142359E-05 & 4.88101505E-04 \\
\hline

\end{tabular}

\end{center}

\

Далее проводится сравнение значений, выдаваемых программой при различном разбиении по $x$ (число слоев по $t$ равно 1000).

Таблица полученных значений для разных разбиений в точке $x=0.2, \ t=0.0025$:

\

\begin{center}
\begin{tabular}{|c|c|c|c|c|}
\hline
n & 10 & 50 & 100 & 250 \\
\hline
exact solution  & \multicolumn{4}{|c|}{1.15398300} \\
\hline
explicit method & 1.15398252 & 1.15398288 & 1.15398347 & 1.15398288 \\
\hline
implicit method & 1.15398383 & 1.15398169 & 1.15397751 & 1.15394521 \\
\hline

\end{tabular}
\end{center}



\end{document}

