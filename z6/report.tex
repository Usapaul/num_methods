\documentclass[12pt,a4paper]{article}
\usepackage[utf8]{inputenc}
\usepackage[english,russian]{babel}
\usepackage[OT1]{fontenc}
\usepackage{amsmath}
\usepackage{amsfonts}
\usepackage{amssymb}
\usepackage{graphicx}
\usepackage{setspace,amsmath}
\usepackage[left=2cm,right=2cm,top=2cm,bottom=2cm]{geometry}
\begin{document}

\begin{flushleft}
\large Усачев Павел, 491 группа
\end{flushleft}

\section*{Решение одномерного уравнения теплопроводности методом сеток}

\

\section{Формулировка задачи}

Уравнение теплопроводности в общем виде:

$$
 \frac{\partial u}{\partial t} =
 a(x,t) \frac{\partial ^2 u}{\partial x^2} +
 b(x,t) \frac{\partial u}{\partial x} +
 c(x,t) u(x,t) + f(x,t),
$$
$$
 x \in [a,b], \ t \in [0,T]
$$

Его мы и хотим решить методом сеток

Начальные условия:

$$
 u(x,0) = \phi(x)
$$

Граничные условия:

$$
 \left[\alpha_1 \frac{\partial u}{\partial x} + \alpha_2 u\right]_{x=a}=\Psi_1(t) 
$$
$$
 \left[\beta_1 \frac{\partial u}{\partial x} + \beta_2 u\right]_{x=b} = \Psi_2(t) 
$$

Шаг сетки по $x$ при заданном $n$: $ \displaystyle h = \frac{b-a}{n}. \ $
$ \displaystyle x_i = a + ih, \ i=0,...,n $

При условии устойчивости:
$$
 \frac{\tau}{h^2} \leq \frac{1}{2A}, \ 
 A = \mathop{\text{max}}_{\Omega} |a(x,t)|, \ 
 \text{где $\Omega$ -- область существования решения $u(x,t)$}
$$

Разбиение по $t$ выглядит следующим образом:

$$
 t_k = k \cdot \tau, \ k=1,..,m, \ 
 \text{$m$ -- количество {\bfseries слоев}, которое нас интересует}
$$

\

Решение на сетке:
$ \displaystyle u_i^k = u(x_i,t_k) $

\section{Условие задачи}

$$
 \frac{\partial u}{\partial t} =
 \frac{\partial ^2 u}{\partial x^2} + \exp(-t)(x^2 - x + 2) 
$$
$$
 x \in [0,1], \ t \in [0,T]
$$

\

Начальные условия: $ \displaystyle u(x,0) = \cos 0.5x + (1-x)x$

\

Граничные условия:
$$ u(0,t) = \exp(-0.25t) $$
$$ u(1,t) = \exp(-0.25t) \cos 0.5 $$

Причем, известно точное решение задачи:
$$
 \displaystyle u^*(x,t) = 
 \exp(-0.25t) \cos 0.5x + \exp(-t) (1-x)x
$$

\section{Вывод формул}



\section{Результаты}

\





\end{document}

